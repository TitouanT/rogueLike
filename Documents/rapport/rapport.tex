\documentclass[12pt]{report}
\usepackage[utf8x]{inputenc}
\usepackage[T1]{fontenc}
\usepackage[francais]{babel}
\usepackage{amsmath}
\usepackage{graphicx}
\usepackage[colorinlistoftodos]{todonotes}

\usepackage[top=3cm, bottom=3cm,left=2.5cm, right=2.5cm]{geometry}

\usepackage{hyperref}
\hypersetup{
    colorlinks,
    citecolor=black,
    filecolor=black,
    linkcolor=black,
    urlcolor=black
}

\begin{document}

\begin{titlepage}
	
	\newcommand{\HRule}{\rule{\linewidth}{0.5mm}}
	\center
	
	\textsc{\LARGE Rapport de projet}\\[1.5cm]
	
	\HRule \\[0.4cm]
	{ \huge \bfseries RogueLike}\\[0.4cm]
	\HRule \\[1.5cm]
	 
	{\large \today}\\[1cm]
	

	Emeric \textsc{Mottier}\\
	Valentin \textsc{Pelloin}\\
	Titouan \textsc{Teyssier}\\
	L2 Sciences pour l'ingénieur
	
	Université du maine\\
	72000 Le Mans

	\large Projet blabla\\[0.5cm]
	
	
	{ \large \bfseries Cfkdsfsd}\\[0.4cm]
	
	\vfill
\end{titlepage}

\tableofcontents

\chapter{Introduction}

\chapter{Organisation}

\chapter{Analyse et Conception}
	\section{Comment jouer ?}
		\begin{itemize}
			\item{Les déplacement :}
				nous pouvons gérer nos déplacements sur la map grace aux fleches de direction. 
			\item{Interactions avec des objets :}	
				les interactions avec un objet (kit de soins, nourriture, escalier) se font avec la touche <<entrée>>.
			\item{Ouvrir et fermer une porte :}	
				si vous souhaitez ouvrir une porte, déplacez-vous devant la porte, appuyer sur la touche <<o>> et marqué la direction de la porte avec les fleches de direction.
				Mais pour fermer, c'est le meme principe que pour ouvrir une porte sauf que la touche est <<c>> au lieu de <<o>>.
			\intem{Gestion de l'inventaire :}
				pour voir votre inventaire, vous devez appuyer sur la touche <<i>>. Pour prendre un objet (kit de soins, nourriture), vous devez appuyez sur la touche <<g>> mais pour poser votre inventaire, vous devez appuyez sur la touche <<d>> et indiquer la case  se trouve l'objet.
		\end{itemize}
		
		
\chapter{Codage, méthode et outil}

\chapter{Résultat et conclusion}

\end{document}