\documentclass[12pt]{report}
\usepackage[utf8x]{inputenc}
\usepackage[T1]{fontenc}
\usepackage[francais]{babel}
\usepackage{amsmath}
\usepackage{graphicx}
\usepackage[colorinlistoftodos]{todonotes}

\usepackage[top=3cm, bottom=3cm,left=2.5cm, right=2.5cm]{geometry}

\usepackage{hyperref}
\hypersetup{
    colorlinks,
    citecolor=black,
    filecolor=black,
    linkcolor=black,
    urlcolor=black
}

\begin{document}

\begin{titlepage}
	
	\topskip0pt
	\newcommand{\HRule}{\rule{\linewidth}{0.5mm}}
	\center
	
	\vspace*{30pt}
	
	\textsc{\Large UFR Sciences et techniques\\ Université du Maine}\\[1.5cm]
	
	\Large RAPPORT DE PROJET\\[1.5cm]
	

	\vspace*{\fill}
	
	%% Titre du rapport de projet %%
	\HRule \\[0.4cm]
	{ \huge \bfseries RogueLike}\\
	\HRule \\[1.5cm]
	 
	 
	 \vspace{12pt}
	 
	{\large \today}\\[1cm]
	
	\vspace{12pt}
	
	\Large 
	
		Emeric \textsc{Mottier}\\
		Valentin \textsc{Pelloin}\\
		Titouan \textsc{Teyssier}\\
	
		\vspace{12pt}
		
		L2 Sciences pour l'ingénieur
		
	\vspace*{\fill}
	
\end{titlepage}

\tableofcontents

\chapter{Introduction}

\chapter{Organisation}

\chapter{Analyse et Conception}
	\section{Comment jouer ?}
		\begin{itemize}
			\item{Les déplacement :}
				nous pouvons gérer nos déplacements sur la map grâce aux flèches de direction. 
			\item{Interactions avec des objets :}	
				les interactions avec un objet (kit de soins, nourriture, escalier) se font avec la touche <<entrée>>.
			\item{Ouvrir et fermer une porte :}	
				si vous souhaitez ouvrir une porte, déplacez-vous devant la porte, appuyer sur la touche <<o>> et marqué la direction de la porte avec les flèches de direction.
				Mais pour fermer, c'est le même principe que pour ouvrir une porte sauf que la touche est <<c>> au lieu de <<o>>.
			\item{Gestion de l'inventaire :}
				pour voir votre inventaire, vous devez appuyer sur la touche <<i>>. Pour prendre un objet (kit de soins, nourriture), vous devez appuyez sur la touche <<g>> mais pour poser votre inventaire, vous devez appuyez sur la touche <<d>> et indiquer la case  se trouve l'objet.
		\end{itemize}
		
		
\chapter{Codage, méthode et outil}

\chapter{Résultat et conclusion}

\end{document}