\documentclass[12pt]{report}
\usepackage[utf8x]{inputenc}
\usepackage[T1]{fontenc}
\usepackage[francais]{babel}
\usepackage{amsmath}
\usepackage{graphicx}
\usepackage[colorinlistoftodos]{todonotes}

\usepackage[top=2cm, bottom=2cm,left=2cm, right=2cm]{geometry}

\usepackage{hyperref}
\hypersetup{
    colorlinks,
    citecolor=black,
    filecolor=black,
    linkcolor=black,
    urlcolor=black
}

\begin{document}

\begin{titlepage}
	
	\topskip0pt
	\newcommand{\HRule}{\rule{\linewidth}{0.5mm}}
	\center
	
	\vspace*{30pt}
	
	\textsc{\Large UFR Sciences et techniques\\ Université du Maine}\\[1.5cm]
	
	\Large RAPPORT DE PROJET\\[1.5cm]
	

	\vspace*{\fill}
	
	%% Titre du rapport de projet %%
	\HRule \\[0.4cm]
	{ \huge \bfseries RogueLike}\\
	\HRule \\[1.5cm]
	 
	 
	 \vspace{12pt}
	 
	{\large \today}\\[1cm]
	
	\vspace{12pt}
	
	\Large 
	
		Emeric \textsc{Mottier}\\
		Valentin \textsc{Pelloin}\\
		Titouan \textsc{Teyssier}\\
	
		\vspace{12pt}
		
		L2 Sciences pour l'ingénieur
		
	\vspace*{\fill}
	
\end{titlepage}

\tableofcontents

\chapter{Introduction}

Nous avons choisi le jeu \emph{Roguelike} car c'est un jeu que nous trouvons intéressant, puisqu'il est complet, et que c'est un jeu aux possibilités infinies :  il est toujours possible d'aj

\chapter{Organisation}

\chapter{Analyse et Conception}
	\section{Comment jouer ?}
		\begin{itemize}
			\item{Lancement du jeu : \\}
				Pour commencer à jouer, vous devez télécharger le jeu à partir de l'adresse suivante : \href{https://github.com/TitouanT/rogueLike/} {rogueLike}; avec la commande suivante : git clone https://github.com/TitouanT/rogueLike/ (sur linux) et git clone git@github.com:TitouanT/rogueLike.git (sur Mac). \\
				Vous faîtes : <<cd rogueLike>>. \\
				Puis vous compilez grâce au makefile : <<make install>>. \\
				Le jeu commence dès que vous faîtes : <<./rogueLike>>		
			\item{Les déplacement : \\}
				Nous pouvons gérer nos déplacements sur la carte grâce aux flèches de direction. 
			\item{Interactions avec des objets : \\}	
				Les interactions avec un objet (serringues de soins, nourriture, escalier) se font avec la touche <<entrée>>.
			\item{Ouvrir et fermer une porte :\\}	
				Si vous souhaitez ouvrir une porte, déplacez-vous devant la porte, appuyer sur la touche <<o>> et marqué la direction de la porte avec les flèches de direction.
				Mais pour fermer, c'est le même principe que pour ouvrir une porte sauf que la touche est <<c>> au lieu de <<o>>.
			\item{Gestion de l'inventaire : \\}
				Pour voir votre inventaire, vous devez appuyer sur la touche <<i>>. Pour prendre un objet (serringues de soins, nourriture), vous devez appuyez sur la touche <<g>> mais pour poser votre inventaire, vous devez appuyez sur la touche <<d>> et indiquer la case se trouve l'objet.
			\item{Sauvegarder sa partie :\\}
				Vous pouvez sauvegarder la partie à tout moment avec la touche <<s>>, cette manoeuvre n'arrêtera pas votre expérience de jeu.		
		\end{itemize}
		
		
\chapter{Codage, méthode et outil}

\chapter{Résultat et conclusion}

\chapter{Annexe}


\end{document}